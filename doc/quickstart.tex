\documentclass[a4paper]{article}

\usepackage{a4wide}

\begin{document}

\section{Introduction}

TransLucid is a programming language in which equations are defined to
vary in an arbitrary dimensional context. A TransLucid instance can run several
instants; at each instant, the user optionally defines new equations, and makes
demands for expressions to be evaluated in the current instant.

\section{Getting Started}

The simplest program to run is the empty program which runs for one instant and
does nothing. Simply run \verb|tltext| on the command line, and type 
\verb|Ctrl-D|.

To print the string \emph{``Hello world!''} to the screen, we can make a 
demand in the first instant to evaluate the string \verb|"Hello world!"|.
Unless otherwise specified, all programs will be run by typing \verb|tltext| on
the command line.
\begin{verbatim}
%%
"Hello world!";;
^D
`Hello world!`
\end{verbatim}

Next, we will define the variable \verb|x|, which will be the constant value
\verb|42| in all contexts, then in the first instant we will make a demand for
the expression \verb|x|.
\begin{verbatim}
var x = 42;;
%%
x;;
^D
42
\end{verbatim}

\end{document}
